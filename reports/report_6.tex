\documentclass{article}
\usepackage[UTF8]{ctex}
\usepackage{float,indentfirst,verbatim,fancyhdr,graphicx,listings,longtable,amsmath, amsfonts,amssymb}

\textheight 23.5cm \textwidth 15.8cm
\topmargin -1.5cm \oddsidemargin 0.3cm \evensidemargin -0.3cm

\title{数值代数实验报告 6}
\author{马天开}

\begin{document}
\maketitle

\section{问题描述}

\subsection{求多项式方程的模最大根}

\subsubsection{}用 C++ 编制利用幂法求多项式方程
\[
    f(x) = x^n + \alpha_{n-1}x^{n-1} +\cdots+ \alpha_1 x + \alpha_0 = 0
\] 的模最大根的通用子程序。

\subsubsection{}利用你所编制的子程序求下列各高次方程的模最大根

\begin{enumerate}
    \item $x^3 + x^2 - 5x + 3 = 0;$
    \item $x^3 - 3x - 1 = 0;$
    \item $x^8 + 101x^7 + 208.01x^6 + 10891.01x^5 + 9802.08x^4 + 79108.9x^3-99902x^2 + 790x-1000 = 0.$
\end{enumerate}

\textbf{要求输出迭代次数,用时和最大根的值(注意正负)}


\subsection{求实矩阵的全部特征值}

\subsubsection{}用 C++ 编制利用隐式 QR 算法 (课本算法 6.4.3) 求一个实矩阵的全部特征值的通用子程序。

\subsubsection{利用你所编制的子程序计算方程}
$$x^{41} + x^3 + 1 = 0 $$

的全部根。

\subsubsection{}
设
\[
    A=\begin{bmatrix}
    \;9.1 & 3.0 & 2.6 & 4.0 \;\\
    \;4.2 & 5.3 & 4.7 & 1.6 \;\\
    \;3.2 & 1.7 & 9.4 & x \;\\
    \;6.1 & 4.9 & 3.5 & 6.2\;
    \end{bmatrix}
\]

求当 $x = 0.9, 1.0, 1.1$ 时 A 的全部特征值,并观察并在报告中叙述分析特征值实部、虚部和模长的变化情况。

\textbf{要求输出迭代次数、用时和所有特征值。}

\section{算法说明}

必须实现的算法有:

\begin{enumerate}
    \item 幂法求模最大根 $\Rightarrow$ \verb|IterationMethod/PowerIteration|
    \item Hessenberg 分解 $\Rightarrow$ \verb|HouseHolderMethod/HessenbergMethod|
    \item 双重步位移的 QR 迭代 $\Rightarrow$ \verb|QRMethod/DoubleStepQRIteration|
    \item 隐式 QR 算法 $\Rightarrow$ \verb|QRMethod/QRMethod|
\end{enumerate}

有一些英文命名并不是很规范,有时间会调整的。

\section{运行结果}

\verbatiminput{data/report_6_output.txt}

\newpage

\end{document}